\section{Einleitung}

\subsection{Motivation neuronaler Netze}
Heutzutage findet man in nahezu allen Bereichen des Lebens Programme vor, sei es in Zahnbürsten, Autos oder in Robotern am Fließband. Mithilfe von Programmen sollen also sich wiederholende Aufgaben vereinfacht oder automatisiert werden. Dafür werden die Programme mit prozedurale oder objektorientierte Methoden entwickelt. Bevor mit der Programmierung angefangen werden kann, muss zuerst ein Modell des Problems entworfen werden. Bei einfachen Anwendungen, wie zum Beispiel einer Zahnbürste ist es recht einfach. Möchte man allerdings ein Modell für eine Mustererkennung 

\subsection{Das menschliche Nervensystem}
Das menschliche Gehirn besteht aus etwa 86 Milliarden Nervenzellen, auch Neuronen genannt, die zum Zentralnervensystem miteinander verschaltet sind (im folgendem Nervensystem). Nervenzellen im Gehirn sind hochspezialisierte Zellen, die sich im Gegensatz zu einfacheren Zellen nicht teilbar sind. Das bedeutet, dass sich das Nervensystem nicht über Zellteilung regenerieren kann. Allerdings schafft es das Gehirn durch die Verknüpfung anderer Nervenzellen den Defekt teilweise oder komplett auszugleichen. Jede Nervenzelle kann mit bis zu 10.000 anderen Nervenzellen in Verbindung stehen. 

Jede Nervenzelle ist zwar im Grundaufbau gleich, kann aber unterschiedliche Formen und Größen haben. Das liegt unter anderem daran, dass sich Nervenzellen weiter spezialisieren. So können die einen Nervenzellen eher für motorische und die anderen eher für sensorische Aufgaben bestimmt sein. Dadurch erreichen Nervenzellen unterschiedliche Geschwindigkeiten, wie sie Informationen an andere Nervenzellen weitergeben. Der Informationsfluss ist dabei immer in die gleiche Richtung. In Abbildung \ref{AufbauNeuron} ist der Aufbau einer Nervenzelle dargestellt.

\begin{figure}[hbt]
	\centering
	\includegraphics[width=0.9\linewidth]{./Bilder/Aufbau_Gehirn_gehirnlernen-de}
	\caption{Aufbau Neuron}
	\label{AufbauNeuron}
\end{figure}

Der hier stark vereinfachte Aufbau einer Nervenzelle besteht aus dem Zellkörper, den Dendriten, dem Axon und den Synapsen. Der Zellkörper bildet die zentrale Einheit einer Nervenzelle und erhält seine Informationen über die Dendriten. Die Dendriten sind baumartig verzweigt und mit anderen Nervenzellen über Synapsen verbunden. Der Zellkörper bildet eine Summation über die Signale der vielen verschieden Dendriten. Dabei kann jedes Dendrit eine erregende oder hemmende Wirkung auf den Zellkörper haben. Bei ausreichender Stärke leitet der Zellkörper das Signal weiter. Wenn eine Erregungsweiterleitung stattfindet, wird das Signal über das Axon und Synapsen an die nächste Nervenzelle weitergeleitet. Ein Axon hat einen Durchmesser von 0,002 - 0,01 Millimetern und kann bis zu einem Meter lang sein. Umso dicker ein Axon ist, desto schneller findet auch die Weiterleitung statt. Jede Nervenzelle kann nur über ihre Synapsen mit anderen Nervenzellen kommunizieren. Dafür hat jede Nervenzelle bis zu 10.000, in manchen Extremfälle sogar mehr als 100.000 Synapsen. Synapsen kommunizieren untereinander mit chemischen oder elektrischen Signalen. Wobei elektrische in chemische und chemische in elektrische Signale umgewandelt werden. In der Regel kommunizieren Synapsen über chemische Signale. Der Grund dafür ist, dass Synapsen meisten keinen direkten Kontakt zueinander haben, sondern ein kleiner Abstand von 20 bis 50 Nanometern bleibt. Bei elektrischen Synapsen ist der Abstand so nah, dass über eine kleine Brücke (gap junction) kommuniziert wird. Dabei kann das Signal schneller weitergeleitet werden.

Das Nervensystem ist ein großes Netz aus sehr vielen Nervenzellen, die unterschiedlich stark miteinander verbunden sind und unterschiedlich schnell Signale weiterleiten. Wenn eine Nervenzelle ein Signal auslöst, weil eine gewisse Schwelle überschritten wurde spricht man auch davon, dass das Aktionspotenzial erreicht wurde. Das Aktionspotenzial kann nur erreicht werden, wenn vorher genügend vorgeschaltete Nervenzellen ein Signal gesendet haben. Der Anfang einer Verkettung von Nervenzellen kann zum Beispiel über sehen, schmecken, spüren oder ähnliches stattfinden. Wird ein Aktionspotenzial in den dafür verantwortlichen Nervenzellen erreicht, werden sie wieder Signale an andere Nervenzellen senden. Am Ende kann das Signal bei Nervenzellen ankommen, die für eine Bewegung verantwortlich sind, wie zum Beispiel eine Gehbewegung. Das ist ein sehr stark vereinfachtes Beispiel und soll nur dazu dienen die Mechanismen eines Nervensystems zu verstehen.

Dieser Text Falls im Text in diesem Abschnitt nicht anders angegeben sind über die Quellen \cite{dasgehirn.info} und \cite{gehirnlernen.de} zu finden.